%%%%%%%%%%%%%%%%%%%%%%%%%%%%%%%%%%%%%%%%%
% University/School Laboratory Report
% LaTeX Template
% Version 3.1 (25/3/14)
%
% This template has been downloaded from:
% http://www.LaTeXTemplates.com
%
% Original author:
% Linux and Unix Users Group at Virginia Tech Wiki 
% (https://vtluug.org/wiki/Example_LaTeX_chem_lab_report)
%
% License:
% CC BY-NC-SA 3.0 (http://creativecommons.org/licenses/by-nc-sa/3.0/)
%
%%%%%%%%%%%%%%%%%%%%%%%%%%%%%%%%%%%%%%%%%

%----------------------------------------------------------------------------------------
%	PACKAGES AND DOCUMENT CONFIGURATIONS
%----------------------------------------------------------------------------------------

\documentclass{article}
\usepackage{float}
\usepackage[version=3]{mhchem} % Package for chemical equation typesetting
\usepackage{siunitx} % Provides the \SI{}{} and \si{} command for typesetting SI units
\usepackage{graphicx} % Required for the inclusion of images
\usepackage{natbib} % Required to change bibliography style to APA
\usepackage{amsmath} % Required for some math elements 
\usepackage{fullpage} %use smaller top and bottom margins
\usepackage[utf8]{inputenc} %TODO hier ev.  latin1 setzen
%\headheight = 20pt
\headsep = 35pt %space between headline and body text
%\voffset = 0pt

\usepackage{listings}
\usepackage{courier}
\usepackage{color}   
\usepackage{titling}
\usepackage{url}

\usepackage{fancyvrb}

\definecolor{dkgreen}{rgb}{0,0.6,0}
\definecolor{gray}{rgb}{0.5,0.5,0.5}
\definecolor{mauve}{rgb}{0.58,0,0.82}

\setlength\parindent{0pt} % Removes all indentation from paragraphs

\renewcommand{\labelenumi}{\alph{enumi}.} % Make numbering in the enumerate environment by letter rather than number (e.g. section 6)

%\usepackage{times} % Uncomment to use the Times New Roman font



\lstset{ %
	language=R,                     % the language of the code
	basicstyle=\footnotesize,       % the size of the fonts that are used for the code
	numbers=left,                   % where to put the line-numbers
	numberstyle=\tiny\color{gray},  % the style that is used for the line-numbers
	stepnumber=1,                   % the step between two line-numbers. If it's 1, each line
	% will be numbered
	numbersep=5pt,                  % how far the line-numbers are from the code
	backgroundcolor=\color{white},  % choose the background color. You must add \usepackage{color}
	showspaces=false,               % show spaces adding particular underscores
	showstringspaces=false,         % underline spaces within strings
	showtabs=false,                 % show tabs within strings adding particular underscores
	frame=single,                   % adds a frame around the code
	rulecolor=\color{black},        % if not set, the frame-color may be changed on line-breaks within not-black text (e.g. commens (green here))
	tabsize=2,                      % sets default tabsize to 2 spaces
	captionpos=b,                   % sets the caption-position to bottom
	breaklines=true,                % sets automatic line breaking
	breakatwhitespace=false,        % sets if automatic breaks should only happen at whitespace
	title=\lstname,                 % show the filename of files included with \lstinputlisting;
	% also try caption instead of title
	keywordstyle=\color{blue},      % keyword style
	commentstyle=\color{dkgreen},   % comment style
	stringstyle=\color{mauve},      % string literal style
	escapeinside={\%*}{*)},         % if you want to add a comment within your code
	morekeywords={*,...}            % if you want to add more keywords to the set
} 



\setlength{\droptitle}{-10em}   % This is your set screw


%----------------------------------------------------------------------------------------
%	DOCUMENT INFORMATION
%----------------------------------------------------------------------------------------

\title{Übung 04 \\ \texttt{Hypothesen testen} \\ Korrelation, Wilcoxon-test und Chi$^{2}$ in \textbf{R} \\ INFI-IS \\ 5xHWII} % TODO: Bitte den Namen eintragen
\author{Albert Greinöcker, Thomas Kefer}
\date{\today} % Date for the report

\begin{document}
	
	\maketitle % Insert the title, author and date
	
	\begin{center}
		
		\includegraphics[width=10cm]{../images/logo.png}
	\end{center}
	\vspace{1cm}
%----------------------------------------------------------------------------------------

\section{Hypothesen testen}


So wie im \texttt{Student Performance} - Datensatz besprochen soll nun ein etwas größerer Fragebogen kennengelernt und teilweise ausgeweitet werden. Ein Download des Datensatzes plus einem PDF in dem genaue Beschreibungen zu den Variablen zu finden sind, befindet sich im Moodle unter: \texttt{european social survey}.


\subsection{Importieren des Datensatzes}

Die entsprechende Datei entpacken und die Datei \texttt{ESS8e02.1\_F1.csv}  so wie im gemeinsamen Skript zum European Social Survey beschrieben, importieren.

 \begin{lstlisting} 
d <- read.csv("../datasets/ESS8e02.1_F1.csv", sep=",", encoding ="UTF-8")
 \end{lstlisting}

Im Datensatz sind die einzelnen Variablen in Zahlen kodiert (so wie es sich gehört). Will man diesen Zahlen Beschriftungen zuordnen, macht man das auf diese Art:

 \begin{lstlisting} [caption={Hier wird z.B. der Variable \texttt{gndr} (Geschlecht) die Werte \texttt{"Male","Female", "No Answer"} für die Zahlen \texttt{1,2,9} zugeordnet.}]
d$gndr <- factor(d$gndr, levels=c(1,2), labels = c("Male","Female"))
d$vote <- factor(d$vote, levels=c(1,2,3,7,8,9), labels = c("Yes","No", "Not eligible to vote", "Refusal", "Don't know", "No answer"))
 \end{lstlisting}



In einzelnen Auswertungen wird der Vergleich zwischen 2 Ländern oder die Situation in nur einem Land untersucht. Aus diesem Grund müssen Subsets mit nur diesen Ländern gebildet werden. Das geht z.B. so: \\

 \begin{lstlisting} [caption={Hier wird 1x ein Datensatz aller Österreicher und 1x ein Datensatz mit Österreichern und Italienern erzeugt.}]
d_at <- d[d$cntry == 'AT',]
d_at_it <- d[d$cntry %in% c('AT','IT'),]
 \end{lstlisting}

Leider kann es dann sein, wenn - wie im Fall der Länder - schon Beschriftungen automatisch zugeordnet wurden, dass unbenutzte Labels (hier: die Länderkürzel für nicht mehr vorhandene Länder) in der Auswertung auftauchen. Diese kann man so löschen:

 \begin{lstlisting}
d_at_it$cntry <- droplevels(d_at_it$cntry) 
\end{lstlisting}


\subsection{Hypothesen Testen}

In den eigentlichen Aufgabenstellungen sollen bestimmte Hypothesen getestet werden. Es soll bei jeder Hypothese...
\begin{enumerate}
	\item eine Aufbereitung der Daten gemacht werden,
	\item eine grafische Veranschaulichung (sofern möglich) gemacht werden,
	\item ein passender Hypothesentest ausgewählt und ausgeführt werden und
	\item eine Interpretation der Ergebnisse erstellt werden.
\end{enumerate}

\newpage

Hier die Hypothesen (in Klammern die betroffenen Variablen):


\begin{enumerate}
	\item Männer haben mehr Glauben in die Polizei (\texttt{rstplc, gender}).
	\begin{lstlisting}
table(d_gndr_2$gndr,d$trstplc)
d_gender_plc <- d[d$trstplc <= 10,]
table(d_gender_plc$gndr,d_gender_plc$trstplc)
boxplot(d_gender_plc$trstplc, col="red")
boxplot(d_gender_plc$trstplc~d_gender_plc$gndr,col="red")

wilcox.test(trstplc~gndr, data=d_gender_plc)
	\end{lstlisting}
	\begin{lstlisting}
		          0    1    2    3    4    5    6    7    8    9   10
		Male    783  422  722 1015 1164 2642 2517 3700 4208 2258 1497
		Female  622  347  696 1144 1320 3083 2866 4158 4711 2499 1684
	\end{lstlisting}
\begin{minipage}{0.42\textwidth}
	\includegraphics[height=6cm]{./Hypothese1a.png}
\end{minipage}
\begin{minipage}{0.42\textwidth}
	\includegraphics[height=6cm]{./Hypothese1b.png}
\end{minipage}
	\begin{Verbatim}[frame=single]
Wilcoxon rank sum test with continuity correction
data:  trstplc by gndr
W = 238346250, p-value = 0.005222
alternative hypothesis: true location shift is not equal to 0
	\end{Verbatim}

\newpage

	\item Es besteht ein negativer Zusammenhang bei "mehr Strom aus nuklearer Energie" und "mehr Strom aus Solarenergie" (\texttt{elgnuc, elgsun}).
	\begin{lstlisting}
		d$elgsun <- factor (d$elgsun, levels=c(1,2,3,4,5))
		d$elgnuc <- factor (d$elgnuc, levels = c(1,2,3,4,5))
		
		d_sun <- d[as.numeric(d$elgsun) <= 5,]
		d_nuc <- d[as.numeric(d$elgnuc) <= 5,]
		
		d_sun$elgsun <- droplevels(d_sun$elgsun)
		d_nuc$elgnuc <- droplevels(d_nuc$elgnuc)
		
		boxplot(d_sun$elgsun~d_nuc$elgnuc, col="green")
		cor.test(as.numeric(d_sun$elgsun),as.numeric(d_nuc$elgnuc), method="kendall")
	\end{lstlisting}
\begin{minipage}{0.40\textwidth}
	\begin{Verbatim}[frame=single]
     1    2    3    4    5
1 1137 1143 2059 3660 9140
2  479 1886 2395 3694 5886
3  400  937 1535 1211 1892
4  295  470  568  565  903
5   46   52   72   88  221
	\end{Verbatim}
\end{minipage}
\begin{minipage}{0.45\textwidth}
	\includegraphics[height=6cm]{./Hypothese2a.png}
\end{minipage}
\begin{Verbatim}[frame=single]
	Kendall's rank correlation tau

data:  as.numeric(d_sun$elgsun) and as.numeric(d_nuc$elgnuc)
z = -36.255, p-value < 2.2e-16
alternative hypothesis: true tau is not equal to 0
sample estimates:
tau 
-0.1536407 
\end{Verbatim}
	

\newpage

	\item In Österreich ist der Eindruck, dass sich der Klimawandel schlecht auf die die Menschen auswirkt, stärker als in Ungarn (\texttt{ccgdbd}). 
	\begin{lstlisting}
		d_at_hu <- d[d$cntry %in% c("AT","HU"),]
		d_at_hu <- d_at_hu[as.numeric(d_at_hu$ccgdbd) <= 10,]
		d_at_hu$cntry <- droplevels(d_at_hu$cntry)
		table(d_at_hu$cntry, d_at_hu$ccgdbd)
		wilcox.test(ccgdbd~cntry, data=d_at_hu)
	\end{lstlisting}
	\begin{Verbatim}[frame=single]
     0   1   2   3   4   5   6   7   8   9  10
AT 209 173 329 361 288 230  89  86  71  20  13
HU 267 183 269 246 177 206  78  55  17   3   3
	\end{Verbatim}
	\begin{Verbatim}[frame=single]
    Wilcoxon rank sum test with continuity correction

data:  ccgdbd by cntry
W = 1579664, p-value = 3.86e-10
alternative hypothesis: true location shift is not equal to 0
	\end{Verbatim}
	\newpage
	
	\item Frauen stimmen einem bedingungslosen Grundeinkommen eher zu (\texttt{basinc}).
	\begin{lstlisting}
		d_basicnc <- d[as.numeric(d$basinc) <=7,]
		d_basicnc$basicnc <- droplevels(d_basicnc$basicnc)
		table(d_basicnc$gndr,d_basicnc$basinc)
		boxplot(d_basicnc$basinc~d_basicnc$gndr, col="blue")
		cor.test(as.numeric(d_basicnc$basinc),as.numeric(d_basicnc$gndr),method="kendall")
	\end{lstlisting}
	\begin{Verbatim}[frame=single]
          1    2    3    4    7
Male   2527 6527 8690 1815  105
Female 2361 7165 9569 1934  130
	\end{Verbatim}
\end{enumerate}
\newpage

Bitte noch 2 Hypothesen selbst wählen und überprüfen!

\end{document}