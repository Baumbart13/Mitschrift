%%%%%%%%%%%%%%%%%%%%%%%%%%%%%%%%%%%%%%%%%
% University/School Laboratory Report
% LaTeX Template
% Version 3.1 (25/3/14)
%
% This template has been downloaded from:
% http://www.LaTeXTemplates.com
%
% Original author:
% Linux and Unix Users Group at Virginia Tech Wiki 
% (https://vtluug.org/wiki/Example_LaTeX_chem_lab_report)
%
% License:
% CC BY-NC-SA 3.0 (http://creativecommons.org/licenses/by-nc-sa/3.0/)
%
%%%%%%%%%%%%%%%%%%%%%%%%%%%%%%%%%%%%%%%%%

%----------------------------------------------------------------------------------------
%	PACKAGES AND DOCUMENT CONFIGURATIONS
%----------------------------------------------------------------------------------------

\documentclass{article}
\usepackage{float}
\usepackage[version=3]{mhchem} % Package for chemical equation typesetting
\usepackage{siunitx} % Provides the \SI{}{} and \si{} command for typesetting SI units
\usepackage{graphicx} % Required for the inclusion of images
\usepackage{natbib} % Required to change bibliography style to APA
\usepackage{amsmath} % Required for some math elements 
\usepackage{fullpage} %use smaller top and bottom margins
\usepackage[utf8]{inputenc} %TODO hier ev.  latin1 setzen
%\headheight = 20pt
\headsep = 35pt %space between headline and body text
%\voffset = 0pt

\usepackage{listings}
\usepackage{courier}
\usepackage{color}   
\usepackage{titling}
\usepackage{url}

\definecolor{dkgreen}{rgb}{0,0.6,0}
\definecolor{gray}{rgb}{0.5,0.5,0.5}
\definecolor{mauve}{rgb}{0.58,0,0.82}

\setlength\parindent{0pt} % Removes all indentation from paragraphs

\renewcommand{\labelenumi}{\alph{enumi}.} % Make numbering in the enumerate environment by letter rather than number (e.g. section 6)

%\usepackage{times} % Uncomment to use the Times New Roman font



\lstset{ %
	language=R,                     % the language of the code
	basicstyle=\footnotesize,       % the size of the fonts that are used for the code
	numbers=left,                   % where to put the line-numbers
	numberstyle=\tiny\color{gray},  % the style that is used for the line-numbers
	stepnumber=1,                   % the step between two line-numbers. If it's 1, each line
	% will be numbered
	numbersep=5pt,                  % how far the line-numbers are from the code
	backgroundcolor=\color{white},  % choose the background color. You must add \usepackage{color}
	showspaces=false,               % show spaces adding particular underscores
	showstringspaces=false,         % underline spaces within strings
	showtabs=false,                 % show tabs within strings adding particular underscores
	frame=single,                   % adds a frame around the code
	rulecolor=\color{black},        % if not set, the frame-color may be changed on line-breaks within not-black text (e.g. commens (green here))
	tabsize=2,                      % sets default tabsize to 2 spaces
	captionpos=b,                   % sets the caption-position to bottom
	breaklines=true,                % sets automatic line breaking
	breakatwhitespace=false,        % sets if automatic breaks should only happen at whitespace
	title=\lstname,                 % show the filename of files included with \lstinputlisting;
	% also try caption instead of title
	keywordstyle=\color{blue},      % keyword style
	commentstyle=\color{dkgreen},   % comment style
	stringstyle=\color{mauve},      % string literal style
	escapeinside={\%*}{*)},         % if you want to add a comment within your code
	morekeywords={*,...}            % if you want to add more keywords to the set
} 



\setlength{\droptitle}{-10em}   % This is your set screw


%----------------------------------------------------------------------------------------
%	DOCUMENT INFORMATION
%----------------------------------------------------------------------------------------

\title{Übung 07 \\ \texttt{Interaktive statistische Webanwendung} \\ Erste Anwendung von mit \texttt{R-shiny} \\ INFI-IS \\ 5xHWII} 
\author{Der Name} % TODO: Bitte den Namen eintragen
\date{\today} % Date for the report

\begin{document}

\maketitle % Insert the title, author and date

\begin{center}
\begin{tabular}{l r}
Abgabetermin: &  lt. mündlicher Vereinbarung \\
%Zusammenarbeit mit: & James Smith \\ % Partner names & Mary Smith \\
Übungsleiter: & Albert Greinöcker % Instructor/supervisor
\end{tabular} \\ 
\vspace{1cm}
\includegraphics[width=2cm]{logo.png}
\end{center}

%----------------------------------------------------------------------------------------

\section{Grundlegende Aufgabenstellung}

Es soll eine Anwendung (im selben Stil wie die gemeinsame Anwendung zur Bevölkerung in Tirol - R-Skripte dazu liegen in Moodle) entstehen). \\

Folgendes soll die Anwendung beinhalten:

\begin{enumerate}
	\item Zumindest 3 Auswahlfelder
	\item Zumindest 3 Grafiken
	\item Eine Tabellendarstellung der Daten (\texttt{DT} -  also eine Datatable)
\end{enumerate}



\subsection{Datensätze}

Diesmal soll allerdings der Datensatz selbst gewählt werden. Ein Umfangreiches Angebot hat hier \texttt{kaggle.com}. Bitte dort einfach mal registrieren und ein paar gezielte Suchen vornehmen. \\

Hier ein paar Beispiele für passende Datensätze:

\begin{enumerate}
	\item Weltweite Daten zum Corona-Virus, der sich mittels Python-Skript immer aktualisieren lässt. \\ \url{https://www.kaggle.com/gauravduttakiit/covid-19}
	\item Spotify-Datensatz von Songs mit Bewertung und akustischen Parametern von 1920-2004 \\
	\url{https://www.kaggle.com/yamaerenay/spotify-dataset-19212020-160k-tracks} 
	\item Der bereits bekannte FIFA19-Datensatz \\
	\url{https://www.kaggle.com/karangadiya/fifa19}
\end{enumerate}

Es müssen nicht unbedingt Datensätze von kaggle, sein, jeder andere passende Datensatz geht natürlich auch ok (z.B. Daten der Diplomarbeit)

\section{Installation auf \texttt{shinyapps.io}}

In Moodle ist ein Link auf eine Anleitung für die Installation zu finden. Bitte die erstellte Anwendung auch online bringen.

\end{document}